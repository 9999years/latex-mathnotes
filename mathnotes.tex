\documentclass{ltxguidex}

\errorcontextlines=10
\usepackage{longtable}
\usepackage{changepage}
\usepackage{metalogo}
\newcommand{\mn}{\ctan{mathnotes}}

\makeatletter
\renewcommand{\SX@explpreset}{
	language=[LaTeX]TeX,
	numbers=none,
}
\makeatother

%\setmainfont{Charter}
\lstset{basewidth=0.6em}

\usepackage[
  nomaketitle,
  notitles,
  noxcolor,
  noenumitem,
  notheorems,
]{mathnotes}

\usepackage{FiraSans}
\usepackage{FiraMono}

\NewDocumentEnvironment{ctandescription}{}
	{\NewDocumentCommand{\pkg}{m}{\item[\ctan{##1}]}
	\begin{description}}
	{\end{description}}

\author{Rebecca Turner\thanks{\email{rbt@sent.as}, \https{becca.ooo}}}
\title{The \mn\ Package}
\date{${VERSION}$}
\begin{document}
\maketitle

\begin{abstract}
	Rebecca Turner's personal macros for typesetting mathematics notes.
\end{abstract}

\begin{note}
	Browse the sources, contribute, or complain at \\
	\https{github.com/9999years/latex-mathnotes}
\end{note}

\tableofcontents
\vfill
\pagebreak

\section{Package options}

Some options are enabled by default, and can be disabled by passing
|no<option>|. The enabled-by-default options are:
\begin{keys}
	\key{fonts}[\bool][true]
  \key{stix}[\bool][true]
    \begin{itemize}
      \item In \XeTeX\ or \LuaTeX, loads the \ctan{unicode-math} package, and
        then, if the \option{stix} option is also given, loads the
        \texttt{STIX2Text} and \texttt{STIX2Math} \textsc{otf} fonts (available
        in the \ctan{stix2-otf} package).

        The \texttt{STIX2Math} font is loaded with stylistic sets 1 (roundhand
        script forms instead of chancery script for |\mathcal| Script
        Alphanumeric Symbols) and 8 (upright, rather than slanted, forms for
        integrals); see ``Stylistic Sets'' in
        \href{http://mirrors.ctan.org/fonts/stix2-otf/STIXTwoMath-Regular.pdf}{the
        \texttt{stix2-otf} documentation} (pp.~51--53) for more information.

      \item In other engines, if the \option{stix} option is also given, loads
        the \package{stix2} package from \ctan{stix2-type1}.
    \end{itemize}

  \key{symbols}[\bool][true]

    Defines a collection of ``natural language'' math-mode symbol commands.
    Most commands are declared with |\ProvideDocumentCommand| so that they
    won't overwrite custom commands you've already defined.

    Note that symbols may look different depending on the \option{fonts} and
    \option{stix} options.

    \begin{longtable}{rl}
      |\lnot| & $\lnot$ \\
      |\Rational|, |\Rat|, |\Q| & $\Rational$ \\
      |\Natural|, |\Nat|, |\N| & $\Natural$ \\
      |\Integer|, |\Int|, |\Z| & $\Integer$ \\
      |\Complex|, |\Comp|, |\C| & $\Complex$ \\
      |\Real|, |\R| & $\R$ \\
      |\powerset| & $\powerset$ \\
      |\vec{A}| & $\vec{A}$ \\
      |\intersection|, |\inter| & $\intersection$ \\
      |\bigintersection|, |\biginter| & $\bigintersection$ \\
      |\union| & $\union$ \\
      |\bigunion| & $\bigunion$ \\
      |\divisible|, |\div| & $a \divisible b$ \\
      |\notdivisible|, |\ndivisible|, |\notdiv|, |\ndiv| & $a \notdivisible b$ \\
      |\floor{A}| & $\floor{A}$ \\
      |\ceil{A}| & $\ceil{A}$ \\
      |\emptyset|, |\es| & $\emptyset$ \\
      |\after| & $g \after f$ \\
      |\cross| & $a \cross b$ \\
      |\img| & $\img f$ \\
      |\pre| & $\pre f$ \\
      |\Stab| & $\Stab f$ \\
      |\FixPt| & $\FixPt f$ \\
      |\id| & $\id$ \\
      |\injection|, |\inj| & $\injection$ \\
      |\surjection|, |\surj| & $\surjection$ \\
      |\bijection|, |\bij| & $\bijection$ \\
      |\restriction|, |\restr| & $f \restr_{\N}$ \\
      |\dd[y]{x}| & $\dd[y]{x}$ \\
      |\pd[y]{x}| & $\pd[y]{x}$ \\
      |\curl| & $\curl$ \\
      |\dive| & $\dive$ \\
    \end{longtable}

  \key{maketitle}[\bool][true]
  \key{titles}[\bool][true]
  \key{xcolor}[\bool][true]
    Load the \ctan{xcolor} package.
  \key{theorems}[\bool][true]
  \key{enumitem}[\bool][true]
\end{keys}

The other options are not enabled by default, and can be enabled by passing
|<option>| --- the option name --- as a package option:
\begin{keys}
	\key{listings}[\bool]
  \key{knowledge}\bool]
  \key{index}\bool]
  \key{footnotes}\bool]
  \key{figures}\bool]
  \key{tabu}\bool]
  \key{kindle}\bool]
\end{keys}

\section{Commands}

\begin{desc}
|\numberthis|
\end{desc}

At the end of a line (before the |\\|) in an \ctan{amsmath} starred
environment, gives an equation a number.

\begin{LTXexample}
\begin{alignat*}{2}
  x &= y \\
  y &= 2z \numberthis \\
  z &= 1/w
\end{alignat*}
\end{LTXexample}

\begin{desc}
|\labelthis{<label>}|
\end{desc}

At the end of a line (before the |\\|) in an \ctan{amsmath} starred
environment, gives an equation a number and label for referencing.

\begin{LTXexample}
\begin{alignat*}{2}
  x &= y \\
  y &= 2z \labelthis{eq:cool} \\
  z &= 1/w.
\end{alignat*}
As we saw in
Equation~\ref{eq:cool}, \dots
\end{LTXexample}

\section{Packages loaded}

\mn\ loads \ctan{amsmath}, \ctan{ntheorem}.

\end{document}
