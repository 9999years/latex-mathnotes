%${LICENSE}$
\documentclass[twocolumn, noxcolor, maketitle]{rbt-mathnotes-hw}
\mathnotes{
  instructor  = Prof.~Ruth Charney ,
  name        = Rebecca Turner ,
  email       = rebeccaturner@brandeis.edu ,
  course      = \textsc{math} 104a (Intro to Topology) ,
  institution = Brandeis University ,
  semester    = Spring 2020 ,
}
\title{Homework 1}
\date{2020-01-18}
\def\T{\mathcal{T}}
\def\basis{\mathcal{B}}
\usepackage[
  letterpaper,
  margin = 1in,
]{geometry}
\raggedbottom
\begin{document}
\maketitle

\section{Topological Spaces}
\subsection{Open Sets and the Definition of a Topology}
\begin{problem}[1.7]
  Define a topology on $\R$ (by listing the open sets within it) that contains
  the open sets $(0,2)$ and $(1,3)$ that contains as few open sets as possible.
\end{problem}
$\T = \{ \emptyset, (0,2), (1,3), (1,2), (0,3), \R \}$.

\subsection{Basis for a Topology}
\begin{problem}[1.10]
  Show that $\mathcal{B} = \{[a,b) \subset \R : a < b\}$ is a basis for a
  topology on $\R$.
\end{problem}
\begin{enumerate}
  \item \textbf{$\emptyset \in \T$, $\R \in \T$.} $\emptyset \in \T$ (by the
    definition of the completion of a basis to a topology).

    Next, we show $\R \in \T$. For all $n \in \Z_{\ge 0}$, $[n-1, n) \in \basis$
    and $[-n + 1, -n) \in \basis$. We know that if $b_1, b_2 \in \basis$, $b_1
    \union b_2 \in \T$, so these short intervals can be gathered together (``unionized'')
    to produce $\R$:
    \[ \bigunion_{n=1}^{\infty} \left( [n-1, n) \union [-n + 1, -n] \right) = \R, \]
    so $\R \in \T$.

  \item \textbf{$\T$ contains all finite intersections of elements of $\T$.}
    Suppose we have two intervals $[a, b)$ and $[c, d)$. Then, we define
    \begin{alignat*}{1}
      a' &= \max(a, c) \\
      b' &= \min(b, d).
    \end{alignat*}
    If $a' > b'$, the intersection $[a, b) \inter [c, d) = \emptyset$, which is
    in $\T$. Otherwise, the intersection is $[a', b')$, which is an element of
    $\basis$. All elements of the basis are in $\T$, so the intersection of two
    elements is in the topology.

    Thankfully, the intersection is itself always a basis element, so we can
    use the same process to show that finite intersections are in $\T$ by
    induction.

  \item \textbf{Unions of elements of $\T$ are in $\T$.} By the definition of
    the completion of a basis to a topology, this is true (all unions of basis
    elements are included in $\T$).
\end{enumerate}

\begin{problem}[1.12]
  % See example 1.9 for defn of \R_l
  Determine which of the following are open sets in $\R_l$. In each case, prove
  your assertion.
  \[ A = [4,5)
    \quad B = \{3\}
    \quad C = [1,2]
    \quad D = (7,8) \]
\end{problem}
\begin{enumerate}
  \item $A$ is open in $\R_l$; $[4,5) \in \basis$.
  \item $B$ is not an open set in $\R_l$; there is no $[a, b) \subset \R$ where
    both $b > a$ and $|[a, b)| = 1$ (because $[0,1) \cong \R$, i.e.~all
    intervals contain infinitely many points).

    (Where $\cong$ means ``is isomorphic to.'')
  \item $C$ is not open in $\R_l$ because the upper bound of an open set in
    $\R_l$ is never inclusive. There is no set of intervals $[a_1, b_1), \dots$
    where the union or intersection of the intervals has an inclusive upper bound.
  \item $D$ is open because we can take
    \[ D = \lim_{n\to\infty} \left[7 + \frac{1}{n}, 8\right), \]
    where $[7 + 1/n, 8) \in \basis$ for any $n \in \R$ with $n \ne 0$.
\end{enumerate}

\pagebreak
\begin{problem}[1.15]
  An arithmetic progression in $\Z$ is a set
  \[ A_{a,b} = \{\dots, a - 2b, a - b, a, a + b, a + 2b, \dots\} \]
  with $a,b \in \Z$ and $b \ne 0$. Prove that the collection of arithmetic
  progressions
  \[ \mathcal{A} = \{ A_{a,b} : a, b \in \Z \text{ and } b \ne 0 \} \]
  is a basis for a topology on $\Z$. The resulting topology is called the
  arithmetic progression topology on $\Z$.
\end{problem}

\begin{proof}
  \def\Ar#1#2{A_{#1,#2}}
  \def\Ars{\mathcal{A}}
  Let us describe the \emph{minimal form} of an arithmetic progression $\Ar ab$
  to be the progression $\Ar{a'}{b'} = \Ar ab$ with $a', b' > 0$ and the
  smallest possible $a'$; in particular, that $a' < b'$.

  We can obtain the minimal form of the progression like so:
  \begin{alignat*}{1}
    a' &= a \bmod b \\
    b' &= |b|, \\
    \Ar{a'}{b'} &= \Ar ab.
  \end{alignat*}

  \begin{remark}
    Two arithmetic progressions have the same elements if their minimal forms are
    the same; this give an equivalence relation on $\Ars$.
  \end{remark}

  Now, suppose we have two arithmetic progressions $\Ar ab$ and $\Ar cd$. We
  assume that the progressions are in minimal form without loss of generality. We
  also assume that $b \le d$ (by swapping $(a,b)$ with $(c,d)$ if necessary),
  again without loss of generality.

  If $b \mid d$ and $a = c$, we have $\Ar ab \subset \Ar cd$. In particular, $\Ar
  ab \inter \Ar cd = \Ar cd$.

  If $b \mid d$ and $a \ne c$, we have $\Ar ab \inter \Ar cd = \emptyset$.

  If $b \nmid d$, we have a different progression. An intersection is generated by
  an index $(n_1, n_2)$, where
  \begin{alignat*}{1}
    a + b n_1 &= c + d n_2. \\
    \intertext{We can then solve for $n_1$:}
    t(n) &= c - a + dn \\
    n_1 &= \frac{t(n_2)}{b}. \\
    \intertext{Next, we have an infinite \emph{set} of possibilities for $n_2$:}
    n_2 &\in \left\{n \in \Z : t(n) \mid b \right\}.
    \intertext{Sorting the possible values of $n_2$ by absolute value, let us call
      the smallest two values $i_1$ and $i_2$. Then, the difference between
      adjacent elements in the intersection progression $\Ar ab \inter \Ar cd$
      is $i_2 - i_1$.
      \endgraf
      Let}
    a' &= a + bi_1 \\
    b' &= i_2 - i_1 \\
    \Ar ab \inter \Ar cd &= \Ar{a'}{b'}.
  \end{alignat*}
  This isn't super rigorous, admittedly (we're missing some inductive reasoning
  about the integers to prove that there are an infinite set of valid values of
  $n_2$, in particular), but I have some fairly convincing Haskell code. And the
  missing steps are mostly boilerplate, and it's late at night already\dots

  In all cases, the intersection of two arithmetic progressions is either empty or
  another arithmetic progression (i.e.~either the empty set or another basis
  element), so the same argument given above for $\R_l$ holds (namely that we can
  extend this to all finite intersections of elements of $\Ars$ inductively).

  Therefore, finite intersections are in the basis. Unions are in the completion
  of the basis (again by definition). The special element $\emptyset$ is in the
  completion (by definition), and $\Z = \Ar01$, so $\Z \in \basis$. Therefore,
  $\Ars$ forms the basis of a topology on $\Z$.
\end{proof}

\pagebreak
\subsection{Closed Sets}
\begin{problem}[1.27(a)]
  The infinite comb $C$ is the subset of the plane illustrated in Figure~1.17
  and defined by
  \begin{multline*}
    C = \{(x,0) : 0 \le x \le 1\} \;\union \\
    \bigg\{ \left( \frac{1}{2^n}, y \right) : n = 0,1,2, \dots \\
      \text{ and } 0 \le y \le 1 \bigg\}.   
  \end{multline*}
  Prove that $C$ is not closed in the standard topology on $\R^2$.
\end{problem}
\begin{proof}
  Suppose $C$ is closed in the standard topology on $\R^2$. Then, its complement
  $C^c = \R^2 \setminus C$ must be an open set.

  The point $(0, 1)$ is not in $C$, so $(0, 1) \in C^c$. Every open ball in $\R^2$
  containing $(0, 1)$ also contains a smaller open ball centered about $(0, 1)$.
  (For example, the open ball about $(-1, 1)$ of radius $1.1$ contains the open
  ball centered about $(0, 1)$ of radius $0.1$.)

  However, every open ball centered about $(0, 1)$ contains infinitely many
  points of $C$; if the ball has radius $r$, all the comb's ``tines'' at $x =
  1/2^n$ for $n > - \log_2 r$ intersect with the ball.

  Therefore, every open ball containing $(0, 1)$ also contains points in $C$. As
  a result, $C^c$ is not open, which contradicts our assumption. Therefore, $C$
  is not closed.
\end{proof}

\begin{problem}[1.32]
  Prove that intervals of the form $[a, b)$ are closed in the lower limit
  topology on $\R$.
\end{problem}
\begin{proof}
  Take some interval $[a, b)$. Its complement is given by $(-\infty, a) \union
  [b, \infty)$. Given that
  \begin{alignat*}{1}
    (-\infty, a) &= \bigunion_{n=1}^\infty [a-n, a) \\
    [b, \infty) &= \bigunion_{n=1}^\infty [b, b+n),
  \end{alignat*}
  the complement of $[a, b)$ is the union of a number of lower-limit intervals
  in $\R$, i.e.~the basis elements. The basis elements and its unions are open
  sets, so the complement of $[a, b)$ is an open set. Then, by the definition of
  a closed set, $[a, b)$ is closed in $\R_l$.
\end{proof}


\end{document}
