%${LICENSE}$
\documentclass{rbt-mathnotes-formula-sheet}
\usepackage{nicefrac}
\ExplSyntaxOn
\NewDocumentCommand \normalized { m }
  { \frac { #1 } { \| #1 \| } }
\let \gr \grad
\def \ddx { \frac{d}{dx} }
% VL = vector literal
\NewDocumentCommand \vl { m } { \left\langle #1 \right\rangle }
\ExplSyntaxOff

\title{Formula Sheet}
\author{Rebecca Turner}
\date{2019-11-12}

% "The most common size for index cards in North America and UK is 3 by 5
% inches (76.2 by 127.0 mm), hence the common name 3-by-5 card. Other sizes
% widely available include 4 by 6 inches (101.6 by 152.4 mm), 5 by 8 inches
% (127.0 by 203.2 mm) and ISO-size A7 (74 by 105 mm or 2.9 by 4.1 in)."
\mathnotes{
  height = 4in ,
  width = 6in ,
}
\begin{document}
\maketitle
\begin{gather*}
% 12.2: Vectors
% 12.3: Dot product
  \textstyle\vec a \cdot \vec b = \sum_i a_i b_i = |\vec a| |\vec b| \cos \theta. \\
% 12.4: Cross product
  \vec a \times \vec b
  % = \left| \begin{array}{rrr}
    % \hat{i} & \hat{j} & \hat{k} \\
    % a_1 & a_2 & a_3 \\
    % b_1 & b_2 & b_3 \\
  % \end{array} \right| \\
  = \langle a_2 b_3 - a_3 b_2,
    \quad a_3b_1 - a_1b_3, \\
    a_1b_2 - a_2b_1 \rangle.\quad
  |\vec a \times \vec b| = |\vec a| |\vec b| \sin \theta.
% 12.5: Equations of lines and planes.
\shortintertext{Param.\ eqns.\ of line through $\langle x_0,y_0,z_0 \rangle$
par.\ to $\langle a,b,c \rangle$:}
  x = x_0 + at,
  \quad y = y_0 + bt,
  \quad z = z_0 + ct. \\
\text{Symm.\ eqns.: }
  \frac{x-x_0}{a}
  = \frac{y-y_0}{b}
  = \frac{z-z_0}{c}. \\
\shortintertext{Vec.\ eqn.\ of plane through $\vec r$ with $\vec n$ normal:}
  \vec n \cdot (\vec r - \vec r_0) = 0,
  \quad \vec n \cdot \vec r = \vec n \cdot \vec r_0. \\
% 13.1: Vector functions
% 13.2: Derivatives/integrals of vector functions
% 13.3: Arc length and curvature
\shortintertext{Length along a vec.\ fn.\ $\vec r(t)$:}
  \textstyle\int_a^b \left|\vec r'(t)\right|\,dt = \int_a^b \sqrt{\sum_i
  r_i'(t)^2}\,dt, \\
\shortintertext{Unit tang.\ $\vec T(t) = \vec r'(t)/\left|\vec
r'(t)\right|$, so curvature of $\vec r(t)$ w/r/t the arc len.\ fn. $s$:}
  \kappa = \left|\frac{d\vec T}{ds}\right|
  = \frac{\left| \vec T'(t) \right|}{\left| \vec r'(t) \right|}
  = \frac{\left| \vec r'(t) \times \vec r''(t) \right|}{\left| \vec r'(t)
  \right|^3}. \\
\text{Unit normal:}\quad
  \vec N(t) = \vec T'(t)\,/\,\left| \vec T'(t) \right| \\
% 14.1: Functions of several variables
% 14.2: Limits and continuity
% 14.3: Partial derivatives
\text{Clairaut's thm.:}\quad
  f_{xy}(a,b) = f_{yx}(a,b) \\
% 14.4: Tangent planes & linear approximations
\shortintertext{Tan.\ plane to $z = f(x,y)$ at $\langle x_0, y_0,
z_0\rangle$:}
  z - z_0 = f_x(x_0, y_0) (x-x_0) \\
    + f_y(x_0, y_0) (y-y_0). \\
% Partial derivatives of f for each variable exist near a point and are
% continuous => f is differentiable at the po\int.
% 14.6: Directional derivatives and the gradient vector
\text{Grad.:}\quad
  \grad f(x,y) = \pd[f]x \hat{i} + \pd[f]y \hat{j}. \\
\shortintertext{Dir.\ deriv.\ towards $\vec u$ at $\langle x_0, y_0 \rangle$:}
  D_{\langle a,b\rangle} f(x_0, y_0) = f_x(x,y) a + f_y(x,y) b \\
  = \grad f(x,y) \cdot \vec u. \\
\shortintertext{Max of $D_{\vec u} f(\vec x) = \left|\grad f(\vec
x)\right|$. Tan.\ plane of $f$ at $\vec p$:}
  0 =
  f_x(\vec p)(x-\vec p_x)
  + f_y(\vec p)(y-\vec p_y) \\
  + f_z(\vec p)(z-\vec p_z).
% 14.7: Maximum and minimum values
\shortintertext{If $f$ has loc.\ extrem.\ at $\vec p$, then $f_x(\vec p) =
0$ (\& $f_y$, etc). If so, let}
  D = \left| \begin{array}{ll}
    f_{xx} & f_{xy} \\
    f_{yx} & f_{yy}
  \end{array}\right|
  = f_{xx} f_{yy} - (f_{xy})^2.
\shortintertext{%
  $D = 0$: no information.
  $D < 0$: saddle pt.
  $D > 0$: $f_{xx}(\vec p) > 0 \implies$ loc.\ min;
  $f_{xx}(\vec p) < 0 \implies$ loc.\ max.
  ($D$ is the \textbf{Hessian mat.})
\endgraf
  Set of possible abs. min and max vals of $f$ in reg.\ $D$: $f$ at critical
  pts.\ and extreme vals.\ on the boundary of $D$.
% 14.8: Lagrange multipliers
\endgraf
  Lagrange mults.: extreme vals of $f(\vec p)$ when $g(\vec p) = k$.
  Find all $\vec x, \lambda$ s.t.
}
  \grad f(\vec x) = \lambda \grad g(\vec x),\quad g(\vec x) = k.
\shortintertext{i.e.\ $f_x = \lambda g_x$, etc.}
% 15.1: Double integrals over rectangles
% 15.2: Iterated integrals
% 15.3: Double integrals over general regions
  \iint f(r\cos\theta, r\sin\theta)r\,dr\,d\theta. \\
  A = \iint_D \left(\sqrt{f_x(x,y)^2 + f_y(x,y)^2 + 1}\right) \,dA. \\
\shortintertext{Line int.s}
  \int_C f(x,y)\,ds = \\
    \int_a^b f(x(t), y(t))\sqrt{\left(\pd[x]t\right)^2 + \left(\pd[y]t\right)^2}\,dt \\
\shortintertext{If $C$ is a smooth curve given by $\vec r(t)$ from $a \le t
\le b$,}
  \int_C \grad f \cdot d\vec r = f(\vec r(b)) - f(\vec r(a)) \\
\text{Spherical coords:}\quad
  x = \rho \sin \phi \cos \theta \\
  y = \rho \sin \phi \sin \theta, z = \rho \cos \phi \\
  \curl \vec F = \\ \left< \pd[R]y - \pd[Q]z, \pd[P]z - \pd[R]x, \pd[Q]x -
  \pd[P]y\right>. \\
  \vec F = \langle P,Q,R \rangle,\quad
  \curl \vec F = \grad \times \vec F \\
  \vec F \text{ ``conservative''} \implies \exists f, \vec F = \grad f. \\
  \dive \vec F = \grad \cdot \vec F = \pd[P]x + \pd[Q]y + \pd[R]z. \\
  \curl(\grad f) = \vec 0,\quad \dive \curl \vec F = 0 \\
\shortintertext{If $C$ is a positively-oriented (ccw) closed curve, $D$
is bounded by $C$, and $\vec n$ represents the normal,}
  % \int_C P\,dx + Q\,dy = \iint_D\left( \pd[Q]{x} - \pd[P]{y} \right). \\
  \oint_C \vec F \cdot \vec n\,ds = \iint_D \dive \vec F(x,y)\,dA.
\end{gather*}

\pagebreak
\raggedright Common derivs:
$f(g(x)) \to g'(x) f'(g(x))$,
$b^x \to b^x \ln b$,
$f^{-1}(x) \to 1/f'(f^{-1}(x))$,
$\ln x \to 1/x$,
$\sin x \to \cos x$, $\cos x \to -\sin x$,
$\tan x \to \sec^2 x$,
$\sin^{-1} x \to 1/\sqrt{1-x^2}$,
$\cos^{-1} x \to -(\sin^{-1}x)'$ (etc.),
$\tan^{-1} x \to 1/(1+x^2)$,
$\sec^{-1} x \to 1/(|x|\sqrt{x^2-1})$.

Common ints (don't forget $+C$):
\begin{gather*}
  x^n \to \frac{x^{n + 1}}{n + 1} + C \quad \text{when } n \ne -1 \\
  1/x \to \ln |x| \\
  \tan x \to -\ln(\cos x) \\
  \int uv'\,dx = uv - \int u'v\,dx \quad\text{(Int.\ by parts)} \\
  \int u\,dv = uv-\int v\,du \\
  \int_{g(a)}^{g(b)} f(u)\,du = \int_a^b f(g(x))g'(x)\,dx
  \quad\text{$u$-substitution.}
\intertext{E.x.\ in $\int 2x \cos x^2\,dx$, let $u=x^2$, find $du/dx=2x
\implies du = 2x\,dx$, subs.\ $\int \cos u\,du = \sin u + C = \sin x^2 +
C$.}
  \iint_R f(x,y)\,dA = \int_\alpha^\beta \int_a^b f(r\cos\theta,
  r\sin\theta)r\,dr\,d\theta
\end{gather*}
\begin{itemize}
  \item Integrand contains $a^2-x^2$, let $x = a\sin\theta$ and use $1 -
  \sin^2 \theta = \cos^2 \theta$.
  \item $a^2 + x^2$, let $x = a\tan\theta$, use $1 + \tan^2 \theta = \sec^2
  \theta$.
  \item $x^2 - a^2$, let $x = a\sec\theta$, use $\sec^2\theta - 1 = \tan^2
  \theta$.
\end{itemize}

\begin{gather*}
  \lim_{x \to 0} \sin x/x = 1 \\
  \lim_{x \to 0} (1-\cos x)/x = 0 \\
  \lim_{x \to \infty} x \sin(1/x) = 1 \\
  \lim_{x \to 0} (1+x)^{1/x} = e \\
  \lim_{x \to 0} (e^{ax}-1)/(bx) = a/b \\
  \lim_{x \to 0^+} x^x = 1 \\
  \lim_{x \to 0^+} x^{-n} = \infty \\
  \text{For $0/0$ or $\pm\infty/\infty$,}\quad
  \lim_{x \to c} f(x)/g(x) = \lim_{x \to c} f'(x)/g'(x) \\
  \text{For $g(x)$ cont.\ at $L$,}
  \lim_{x \to c} f(x) = L \implies \lim_{x \to c} g(L)
\end{gather*}

\end{document}
